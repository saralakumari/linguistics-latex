\documentclass[12pt]{article}
\usepackage{xltxtra}
\usepackage{fontspec}
\setmainfont{Times New Roman}
	\setmonofont{Courier New}
\usepackage{array, booktabs, tabu}
% {tabu} environment replaces {tabular} for better control of column width
	\newcolumntype{L}{>{\raggedright\arraybackslash}X}
	\newcolumntype{R}{>{\raggedleft\arraybackslash}X}
	\newcolumntype{C}{>{\center\arraybackslash}X}

\begin{document}

\begin{table}
\tabulinesep=_2pt^2pt
\begin{tabu}[to=\textwidth]{L[.26] l L[.70] <{\strut}}
\toprule
\multicolumn{3}{l}{Factor 1: Intergenerational Language Transmission}\\ \midrule
Degree of endangerment 		& Grade & Speaker Population \\ \midrule
safe					&  5	& The language is used by all ages, from
										children up.\\
unsafe					&  4	& The language is used by some children in 
										all domains; it is used by all 
										children in limited domains.\\
definitively endangered		&  3	& The language is used mostly by the 
										parental generation and up.\\
severely endangered			& 2	& The language is used mostly by the
										grandparental generation and up.\\
critically endangered			& 1 	& The language is used by very few speakers,
										mostly of great-grandparental generation.\\
extinct					& 0	& There are no speakers.\\
\bottomrule
\end{tabu}
\label{factor1}
\end{table}

\begin{table}
\tabulinesep=_2pt^2pt
\begin{tabu}[to=\textwidth]{L[.26] l L[.70] <{\strut}}
\toprule
\multicolumn{3}{l}{Factor 2: Absolute Number of Speakers}\\ \midrule
\multicolumn{3}{l}{\itshape ``It is impossible to establish a hard and fast rule for interpreting absolute numbers,}\\
\multicolumn{3}{l}{\itshape but a small speech community is always at risk.''}\\
\bottomrule
\end{tabu}
\label{factor2}
\end{table}

\begin{table}
\tabulinesep=_2pt^2pt
\begin{tabu}[to=\textwidth]{L[.26] l L[.70] <{\strut}}
\toprule
\multicolumn{3}{l}{Factor 3: Proportion of Speakers within the Total 
					Population}\\ \midrule
Degree of endangerment 	& Grade & Proportion of speakers within the total 
							reference population \\ \midrule
safe				&  5	& All speak the language.\\
unsafe				&  4	& Nearly all speak the language.\\
definitively endangered	&  3	& A majority speak the language.\\
severely endangered		&  2	& A minority speak the language.\\
critically endangered		&  1 	& Very few speak the language.\\
extinct				&  0	& None speak the language.\\
\bottomrule
\end{tabu}
\label{factor3}
\end{table}


\begin{table}
\tabulinesep=_2pt^2pt
\begin{tabu}[to=\textwidth]{L[.26] l L[.70] <{\strut}}
\toprule
\multicolumn{3}{l}{Factor 4: Shifts in Domains of Language Use}\\ \midrule
Degree of endangerment 	& Grade & Domains and functions \\ \midrule
universal use				&  5	& The language is used in all domains and 
						for all functions.\\
multilingual parity		&  4	& Two or more languages may be used in 
						most social domains
						and for most functions.\\
dwindling domains		&  3	& The language is used in home domains
						and for many functions, but the
						dominant language begins to penetrate
						even home domains.\\
limited or formal domains	&  2	& The language is used in limited social
						domains and for several functions.\\
highly limited domains		&  1 	& The language is used only in a very 
						restricted number of domains and for
						very few functions.\\
extinct				&  0	& The language is not used in any domain
						for any function.\\
\bottomrule
\end{tabu}
\label{factor4}
\end{table}

\begin{table}
\tabulinesep=_2pt^2pt
\begin{tabu}[to=\textwidth]{L[.26] l L[.70] <{\strut}}
\toprule
\multicolumn{3}{l}{Factor 5: Response to New Domains and Media}\\ \midrule
Degree of endangerment 	& Grade & Domains and functions \\ \midrule
dynamic			&  5	& The language is used in all new domains.\\
robust/active	&  4	& The language is used in most new domains.\\
receptive		&  3	& The language is used in many new domains.\\
coping			&  2	& The language is used in some new domains.\\
minimal			&  1 	& The language is used in only a few new domains.\\
inactive		&  0	& The language is not used in any new domains.\\
\bottomrule
\end{tabu}
\label{factor5}
\end{table}

\begin{table}
\tabulinesep=_2pt^2pt
\begin{tabu}[to=\textwidth]{L[.15] L[.85] <{\strut}}
\toprule
\multicolumn{2}{l}{Factor 6: Availability of Materials for Language Education
					 and Literacy}\\ \midrule
Grade & Availability of written materials \\ \midrule
5	& There is an established orthography and a literacy tradition with
		grammars, dictionaries, texts, literature and everyday media. Writing
		in the language is used in administration and education.\\
4	& Written materials exist and children may be exposed to the written form
		at school. Writing in the language is not used in administration.\\
3	& Written materials exist and children may be exposed to the written form
		at school. Literacy is not promoted through print media.\\
2	& Written materials exist, but they may only be useful for some members of
		the community; for others they may have a symbolic significance.
		Literacy education in the language is not a part of the school 
		curriculum.\\
1	& A practical orthography is known to the community and some material is
		being written.\\
0	& No orthography is available to the community.\\
\bottomrule
\end{tabu}
\label{factor6}
\end{table}

\begin{table}
\tabulinesep=_2pt^2pt
\begin{tabu}[to=\textwidth]{L[.26] l L[.70] <{\strut}}
\toprule
\multicolumn{3}{l}{Factor 7: Governmental \& Institutional Language Attitudes} \\
\multicolumn{3}{l}{and Policies Including Official Status \& Use}\\ \midrule
Degree of endangerment 	& Grade & Official attitudes towards language\\ 
			\midrule
equal support			&  5	& All languages are protected.\\
differentiated support		&  4	& Minority languages are protected primarily
					as the language of private domains. The use 
					of the language is prestigious.\\
passive assimilation		&  3	& No explicity policy exists for minority
					languages.\\
active assimilation		&  2	& Government encourages assimilation to the 
					dominant language. There is no protection 
					for minority languages.\\
forced assimilation		&  1 	& The dominant language is the sole official
					language, while non-dominant languages
					are neither recognized nor protected.\\
prohibition			&  0	& Minority languages are prohibited.\\
\bottomrule
\end{tabu}
\label{factor7}
\end{table}

\begin{table}
\tabulinesep=_2pt^2pt
\begin{tabu}[to=\textwidth]{L[.15] L[.85] <{\strut}}
\toprule
\multicolumn{2}{l}{Factor 8: Community Members’ Attitudes towards Their Own
					 Language}\\ \midrule
Grade & Community members' attitudes towards language \\ \midrule
5	& \emph{All} members value their language and wish to see it promoted.\\
4	& \emph{Most} members support language maintenance.\\
3	& \emph{Many} members support language maintenance; others are indifferent
		or may even support language loss.\\
2	& \emph{Some} members support language maintenance; others are indifferent
		or may even support language loss.\\
1	& Only \emph{a few} members support language maintenance; others are 
		indifferent or may even support language loss.\\
0	& \emph{No one} cares if the language is lost; all prefer to use the 
		dominant language.\\
\bottomrule
\end{tabu}
\label{factor8}
\end{table}

\begin{table}
\tabulinesep=_2pt^2pt
\begin{tabu}[to=\textwidth]{L[.26] l L[.70] <{\strut}}
\toprule
\multicolumn{3}{l}{Factor 9: Type and Quality of Documentation}\\ \midrule
Nature of documentation & Grade & Language documentation \\ \midrule
superlative		&  5	&  There are comprehensive grammars and 
				dictionaries, extensive texts, and a 
				constant flow of language materials.
				Abundant annotated high-quality audio and
				video recordings exist.\\
good			&  4	&  There is one good grammar and a 
				number of 
				adequate grammars, dictionaries, texts,
				literature and occasionally updated 
				everyday media; adequate annotated 
				high-quality audio and video recordings
								exist.\\
fair			&  3	&  There may be an adequate grammar or 
				sufficient numbers of grammars, 
				dictionaries and texts but no everyday
				media; audio and video recordings of 
				varying quality or degree of annotation 
				may exist.\\
fragmentary		&  2	&  There are some useful grammatical 
				sketches, 
				word-lists and texts useful for limited
				linguistic research but with inadequate 
				coverage. Audio and video recordings of 
				varying quality, with or without any 
				annotation, may exist.\\
inadequate		&  1 	&  There are only a few grammatical 	sketches,
				short word-lists and fragmentary texts.
				Audio and video recordings do not exist,
				are of unusable quality or are completely
				un-annotated.\\
undocumented			&  0	&  No material exists.\\
\bottomrule
\end{tabu}
\label{factor9}
\end{table}

\end{document}